
\let\negmedspace\undefined
\let\negthickspace\undefined
\documentclass[journal]{IEEEtran}
\usepackage[a5paper, margin=10mm, onecolumn]{geometry}
%\usepackage{lmodern} % Ensure lmodern is loaded for pdflatex
\usepackage{tfrupee} % Include tfrupee package

\setlength{\headheight}{1cm} % Set the height of the header box
\setlength{\headsep}{0mm}     % Set the distance between the header box and the top of the text

\usepackage{gvv-book}
\usepackage{gvv}
\usepackage{cite}
\usepackage{amsmath,amssymb,amsfonts,amsthm}
\usepackage{algorithmic}
\usepackage{graphicx}
\usepackage{textcomp}
\usepackage{xcolor}
\usepackage{txfonts}
\usepackage{listings}
\usepackage{enumitem}
\usepackage{mathtools}
\usepackage{gensymb}
\usepackage{comment}
\usepackage[breaklinks=true]{hyperref}
\usepackage{tkz-euclide} 
\usepackage{listings}
% \usepackage{gvv}                                        
\def\inputGnumericTable{}                                 
\usepackage[latin1]{inputenc}                                
\usepackage{color}                                            
\usepackage{array}                                            
\usepackage{longtable}                                       
\usepackage{calc}                                             
\usepackage{multirow}                                         
\usepackage{hhline}                                           
\usepackage{ifthen}                                           
\usepackage{lscape}

\renewcommand{\thefigure}{\theenumi}
\renewcommand{\thetable}{\theenumi}
\setlength{\intextsep}{10pt} % Space between text and floats



\numberwithin{figure}{enumi}
\renewcommand{\thetable}{\theenumi}

% Marks the beginning of the document
\begin{document}
\bibliographystyle{IEEEtran}

\title{2007-MA-35-51}
\author{EE24BTECH11035 - KOTHAPALLI AKHIL}
% \maketitle
% \newpage
% \bigskip
{\let\newpage\relax\maketitle}

\begin{enumerate}
\item Let $f(z) = 2z^2 - 1$. Then the maximum value of $|f(z)|$ on the unit disc $D = \{z \in \mathbb{C}: |z| \leq 1\}$ equals  
\begin{enumerate} 
    \item 1  
    \item 2  
    \item 3  
    \item 4  
\end{enumerate}

\item Let  
\begin{equation}
f(z) = \frac{1}{z^2 - 3z + 2}.
\end{equation}  
Then the coefficient of $\frac{1}{z^2}$ in the Laurent series expansion of $f(z)$ for $|z| > 2$ is
\begin{enumerate}
    \item 0
    \item 1
    \item 3
    \item 5
\end{enumerate}

\item  Let $f: \mathbb{C} \to \mathbb{C}$ be an arbitrary analytic function satisfying $f(0) = 0$ and $f(1) = 2$. Then
\begin{enumerate}
    \item there exists a sequence $\{z_n\}$ such that $|z_n| > n$ and $|f(z_n)| > n$
    \item there exists a sequence $\{z_n\}$ such that $|z_n| > n$ and $|f(z_n)| \leq n$
    \item there exists a bounded sequence $\{z_n\}$ such that $|f(z_n)| > n$
    \item there exists a sequence $\{z_n\}$ such that $z_n \to 0$ and $f(z_n) \to 2$
\end{enumerate}

\item Define $f: \mathbb{C} \to \mathbb{C}$ by
\begin{equation}
f(z) =
\begin{cases}
0, & \text{if } \text{Re}(z) = 0 \text{ or Im}(z) = 0, \\
z, & \text{otherwise}.
\end{cases}
\end{equation}
Then the set of points where $f$ is analytic is
\begin{enumerate}
    \item $\{z : \text{Re}(z) \neq 0 \text{ and } \text{Im}(z) \neq 0\}$
    \item $\{z : \text{Re}(z) \neq 0\}$
    \item $\{z : \text{Re}(z) \neq 0 \text{ or } \text{Im}(z) \neq 0\}$
    \item $\{z : \text{Im}(z) \neq 0\}$
\end{enumerate}
\item Let $U(n)$ be the set of all positive integers less than $n$ and relatively prime to $n$. Then $U(n)$ is a group under multiplication modulo $n$. For $n = 248$, the number of elements in $U(n)$ is
\begin{enumerate}
    \item 60
    \item 120
    \item 180
    \item 240
\end{enumerate}

\item Let $\mathbb{R}[x]$ be the polynomial ring in $x$ with real coefficients and let $I = (x^2 + 1)$ be the ideal generated by the polynomial $x^2 + 1$ in $\mathbb{R}[x]$. Then
\begin{enumerate}
    \item $I$ is a maximal ideal
    \item $I$ is a prime ideal but NOT a maximal ideal
    \item $I$ is NOT a prime ideal
    \item $\mathbb{R}[x]/I$ has zero divisors
\end{enumerate}

\item Consider $\mathbb{Z}_5$ and $\mathbb{Z}_{20}$ as rings modulo 5 and 20, respectively. Then the number of homomorphisms $\phi: \mathbb{Z}_5 \to \mathbb{Z}_{20}$ is
\begin{enumerate}
    \item 1
    \item 2
    \item 4
    \item 5
\end{enumerate}

\item Let $\mathbb{Q}$ be the field of rational numbers and consider $\mathbb{Z}_2$ as a field modulo 2. Let 
\begin{equation}
f(x) = x^3 - 9x^2 + 9x + 3.
\end{equation}
Then $f(x)$ is
\begin{enumerate}
    \item irreducible over $\mathbb{Q}$ but reducible over $\mathbb{Z}_2$
    \item irreducible over both $\mathbb{Q}$ and $\mathbb{Z}_2$
    \item reducible over $\mathbb{Q}$ but irreducible over $\mathbb{Z}_2$
    \item reducible over both $\mathbb{Q}$ and $\mathbb{Z}_2$
\end{enumerate}
\item Consider $\mathbb{Z}_5$ as a field modulo 5 and let

\begin{equation}
f(x) = x^5 + 4x^4 + 4x^3 + 4x^2 + x + 1
\end{equation}

Then the zeros of $f(x)$ over $\mathbb{Z}_5$ are 1 and 3 with respective multiplicity.

\begin{enumerate}
    \item 1 and 4
    \item 2 and 3
    \item 2 and 2
    \item 1 and 2
\end{enumerate}

\item Consider the Hilbert space $\ell^2 = \left\{ x = \{x_n\}; x_n \in \mathbb{R}, \sum_{n=1}^{\infty} x_n^2 < \infty \right\}$. Let

\begin{equation}
E = \left\{ x = \{x_n\}; |x_n| \leq \frac{1}{n} \text{ for all } n \right\}
\end{equation}

be a subset of $\ell^2$. Then:

\begin{enumerate}
    \item $E^\circ = \left\{ x : |x_n| < \frac{1}{n} \text{ for all } n \right\}$
    \item $E^\circ = E$
    \item $E^\circ = \left\{ x : |x_n| < \frac{1}{n} \text{ for all but finitely many } n \right\}$
    \item $E^\circ = \emptyset$
\end{enumerate}

\item Let $X$ and $Y$ be normed linear spaces and let $T: X \to Y$ be a linear map. Then $T$ is continuous if

\begin{enumerate}
    \item $Y$ is finite dimensional
    \item $X$ is finite dimensional
    \item $T$ is one to one
    \item $T$ is onto
\end{enumerate}
\item Let $X$ be a normed linear space and let $E_1, E_2 \subseteq X$. Define

\begin{equation}
E_1 + E_2 = \{ x+y : x \in E_1, y \in E_2 \}.
\end{equation}

Then $E_1 + E_2$ is

\begin{enumerate}
    \item open if $E_1$ or $E_2$ is open
    \item NOT open unless both $E_1$ and $E_2$ are open
    \item closed if $E_1$ or $E_2$ is closed
    \item closed if both $E_1$ and $E_2$ are closed
\end{enumerate}

\item For each $a \in \mathbb{R}$, consider the linear programming problem

\begin{equation}
\text{Max. } z = x_1 + 2x_2 + 3x_3 + 4x_4,
\end{equation}
subject to
\begin{equation}
a x_1 + 2 x_3 \leq 1,\\ \quad x_1 + a x_3 + 3 x_4 \leq 2,\\ \quad x_1, x_2, x_3, x_4 \geq 0.
\end{equation}

Let $S = \{ a \in \mathbb{R}: \text{ the given LP problem has a basic feasible solution} \}$. Then:

\begin{enumerate}
    \item $S = \emptyset$
    \item $S = \mathbb{R}$
    \item $S = (0, \infty)$
    \item $S = (-\infty, 0)$
\end{enumerate}

\item Consider the linear programming problem

\begin{equation}
\text{Max. } z = x_1 + 5x_2 + 3x_3,
\end{equation}
subject to
\begin{equation}
2x_1 - 3x_2 + 5x_3 \leq 3, \quad 3x_1 + 2x_3 \leq 5, \quad x_1, x_2, x_3 \geq 0.
\end{equation}

Then the dual of this LP problem:

\begin{enumerate}
    \item has a feasible solution but does NOT have a basic feasible solution
    \item has a basic feasible solution
    \item has infinite number of feasible solutions
    \item has no feasible solution
\end{enumerate}

\item Consider a transportation problem with two warehouses and two markets. The warehouse capacities are $a_1 = 2$ and $a_2 = 4$, and the market demands are $b_1 = 3$ and $b_2 = 3$. Let $x_{ij}$ be the quantity shipped from warehouse $i$ to market $j$ and $c_{ij}$ be the corresponding unit cost. Suppose that $c_{11} = 1, c_{12} = 1$ and $c_{22} = 2$. Then $(x_{11}, x_{12}, x_{21}, x_{22}) = (2, 0, 1, 3)$ is optimal for every

\begin{enumerate}
    \item $c_{12} \in [2, 3]$
    \item $c_{12} = [0, 3]$
    \item $c_{12} \in [1, 3]$
    \item $c_{12} \in [2, 4]$
\end{enumerate}
\item The smallest degree of the polynomial that interpolates the data

\[
\begin{array}{c|cccccc}
x & -2 & -1 & 0 & 1 & 2 & 3 \\
\hline
f(x) & -58 & -21 & -12 & -13 & -6 & 27 \\
\end{array}
\]

is:

\begin{enumerate}
    \item[(A)] 3
    \item[(B)] 4
    \item[(C)] 5
    \item[(D)] 6
\end{enumerate}

\item Suppose that \( x_0 \) is sufficiently close to 3. Which of the following iterations \( x_{n+1} = g(x_n) \) will converge to the fixed point \( x = 3 \)?

\begin{enumerate}
    \item 
    \begin{equation}
    x_{n+1} = -16 + 6x_n + \frac{3}{x_n}
    \end{equation}
    
    \item 
    \begin{equation}
    x_{n+1} = \sqrt{3 + 2x_n}
    \end{equation}
    
    \item 
    \begin{equation}
    x_{n+1} = \frac{3}{x_n - 2}
    \end{equation}
    
    \item
    \begin{equation}
    x_{n+1} = \frac{x_n^2 - 3}{2}
    \end{equation}
\end{enumerate}



\end{enumerate}




\end{document}
