\documentclass[12pt]{article}
\setlength{\oddsidemargin}{0in}
\setlength{\evensidemargin}{0in}
\setlength{\textwidth}{6.5in}
\setlength{\topmargin}{0in}
\setlength{\textheight}{9in}
\usepackage{amsmath}
\usepackage{amssymb}
\title{\textbf{Eigen value calculation}}
\author{Akhil kothapalli}

\begin{document}
\maketitle
In this report, we will know about Eigenvalues, Various algorithms for finding eigenvalues of matrices, Comparison between different algorithms and their complexities, The code structure for finding eigenvalues.
\section{Introduction}
  Eigen values plays a vital role in understanding the behaviour of Linear Transformations . These are fundamental in many scientific and Engineering fields.\\
  

\section{Eigen values and Eigen vectors}
    \subsection{Overview of Eigenvalue}
       The eigenvalues of a matrix $A \in \mathbf{C}^{n \times n}$ are the n roots of its characteristic polynomial $p(z)=det(zI-A)$. The set of these roots is called the spectrum and is denoted by $\lambda(A)$. If $\lambda(A)={\lambda_1,\dots,\lambda_n}$, then it follows that
     \begin{equation*}
    \det(A) = \lambda_1 \lambda_2 \dots \lambda_n
\end{equation*}
Moreover, Trace of A is sum of eigen values, i.e,
\begin{equation*}
    tr(A)=\lambda_1+\lambda_2+\lambda_3\dots+\lambda_n
\end{equation*}

    \subsection{Overview of Eigenvector}
        If $\lambda\in\lambda(A)$, then the nonzero vectors $x\in\mathbf{C}^n$ that satisfy
        \begin{equation*}
        Ax=\lambda x
        \end{equation*}
        are referred to as eigenvectors. More precisely, $x$ is a right eigenvector for $\lambda$ if $ Ax=\lambda x$ and a left eigenvector if $x^HA=\lambda x^H$. Unless ostherwise stated, "eigenvector" means "right eigenvector".
\section{Different Methods of finding eigen values}
\section*{3.1 Methods}
\[
\begin{array}{|c|c|c|c|c|}
\hline
\textbf{Method} & \textbf{Large Matrices} & \textbf{Non-Symmetric } & \textbf{Sparse} & \textbf{Complexity} \\ \hline
\textbf{Lanczos} & \checkmark & \times & \checkmark & O(nk) \\ \hline
\textbf{Arnoldi} & \checkmark & \checkmark & \checkmark & O(nk) \\ \hline
\textbf{Jacobi} & \times & \times & \checkmark & O(n^3) \\ \hline
\textbf{QR Algorithm} & \checkmark & \checkmark & \checkmark & O(n^3) \\ \hline
\end{array}
\]

\section*{3.2 Comparison}
\[
\begin{array}{|c|c|c|c|}
\hline
\textbf{Method} & \textbf{Applicability} & \textbf{Accuracy} & \textbf{Convergence Rate}  \\ \hline
\textbf{Jacobi Method} & Symmetric matrices & \textit{High (small values)} & Moderate  \\ \hline
\textbf{QR Algorithm} & General matrices & \textit{Very High} & Fast  \\ \hline
\textbf{Lanczos Method} & Large sparse symmetric matrices & \textit{Moderate} & Fast for large n . \\ \hline
\textbf{Arnoldi Method} & General large matrices & \textit{High} & Fast  \\ \hline
\end{array}
\]


\section{Jacobi Method Overview}
\begin{itemize}
    \item Start with a symmetric matrix. check whether the matrix is diagonal or not. If it is diagonal matrix , the elements of principle diagonal are Eigen values of the matrix.
    \item Find the Largest off-diagonal element $A_{pq}$ of the given Non-diagonal symmetric matrix.
    \item compute Rotation Angle $\theta$ i.e.,
    \begin{equation*}
	    \theta=\frac{1}{2}tan^{-1}{{\frac{2A_{pq}}{A_{pp}-A_{qq}}}}
    \end{equation*}
    \item construct a matrix J , which makes rotation in p-q plane.
    \item Update the given matrix A as $A'=J^TAJ$ till we get a diagonal matrix.
\end{itemize}
\subsection{Reason for Choosing Jacobi's Method}
\begin{itemize}
   \item Accuracy, Simplicity.
   \item works well for small to medium sized elements.
    \item  It can sometimes compute tiny eigenvalues and their
eigenvectors with much higher accuracy than other methods. 
   \item Easy to code in C language.
\end{itemize}

\section{Implementation}
       \subsection{Code Structure}
       \begin{itemize}
           \item Contains the functions: matrix multiplications , matrix transposing, checking whether the matrix is diagonal or not,e.t.c.
           \item  In main function, There is a while loop ,For iteration of matrices till the matrix becomes diagonal matrix. 
       \end{itemize}
        
       
\section{Conclusion}
 This report mainly concentrates on Calculation of Eigen values for a Symmetric Matrix using Jacobi's method. Which is an iterative way of rotating a matrix and making the matrix diagonal. Since, jacobi's iterative method is simple and gives high precision for small values it is used more widely for symmetric matrices. 
\section{References}
    Matrix computations,Book by Gene H. Golub\\
    Linear Algebra and its applications ,Gilbert strang

\end{document}

