\let\negmedspace\undefined
\let\negthickspace\undefined
\documentclass[journal]{IEEEtran}
\usepackage[a5paper, margin=10mm, onecolumn]{geometry}
%\usepackage{lmodern} % Ensure lmodern is loaded for pdflatex
\usepackage{tfrupee} % Include tfrupee package

\setlength{\headheight}{1cm} % Set the height of the header box
\setlength{\headsep}{0mm}     % Set the distance between the header box and the top of the text

\usepackage{gvv-book}
\usepackage{gvv}
\usepackage{cite}
\usepackage{amsmath,amssymb,amsfonts,amsthm}
\usepackage{algorithmic}
\usepackage{graphicx}
\usepackage{textcomp}
\usepackage{xcolor}
\usepackage{txfonts}
\usepackage{listings}
\usepackage{enumitem}
\usepackage{mathtools}
\usepackage{gensymb}
\usepackage{comment}
\usepackage[breaklinks=true]{hyperref}
\usepackage{tkz-euclide} 
\usepackage{listings}
% \usepackage{gvv}                                        
\def\inputGnumericTable{}                                 
\usepackage[latin1]{inputenc}                                
\usepackage{color}                                            
\usepackage{array}                                            
\usepackage{longtable}                                       
\usepackage{calc}                                             
\usepackage{multirow}                                         
\usepackage{hhline}                                           
\usepackage{ifthen}                                           
\usepackage{lscape}

\renewcommand{\thefigure}{\theenumi}
\renewcommand{\thetable}{\theenumi}
\setlength{\intextsep}{10pt} % Space between text and floats



\numberwithin{figure}{enumi}
\renewcommand{\thetable}{\theenumi}

% Marks the beginning of the document
\begin{document}
\bibliographystyle{IEEEtran}

\title{09-April-2024 Shift-2}
\author{EE24BTECH11035 - KOTHAPALLI AKHIL}
% \maketitle
% \newpage
% \bigskip
{\let\newpage\relax\maketitle}

\begin{enumerate}
\setcounter{enumi}{15}


    \item Consider the line $L$ passing through the points $(1, 2, 3)$ and $(2, 3, 5)$. The distance of the point 
    \begin{equation*}
    \left( \frac{11}{3}, \frac{11}{3}, \frac{19}{3} \right)
    \end{equation*}
    from the line $L$ along the line 
    \begin{equation*}
    \frac{3x - 11}{2} = \frac{3y - 11}{1} = \frac{3z - 19}{2}
    \end{equation*}
    is equal to:
    \begin{enumerate}
        \item $5$
        \item $4$
        \item $3$
        \item $6$
    \end{enumerate}

    \item Let 
    \begin{equation*}
    B = \begin{bmatrix} 1 & 3 \\ 1 & 5 \end{bmatrix}
    \end{equation*}
    and $A$ be a $2 \times 2$ matrix such that $AB^{-1} = A^{-1}$. If $BCB^{-1} = A$ and 
    \begin{equation*}
    C^4 + \alpha C^2 + \beta I = 0
    \end{equation*}
    then $2\beta - \alpha$ is equal to:
    \begin{enumerate}
        \item $8$
        \item $2$
        \item $16$
        \item $10$
    \end{enumerate}

    \item The area (in square units) of the region enclosed by the ellipse $x^2 + 3y^2 = 18$ in the first quadrant below the line $y = x$ is:
    \begin{enumerate}
        \item ${\sqrt{3}\pi + 1}$
        \item $\sqrt{3}\pi +\frac{3}{4}$
        \item ${\sqrt{3}\pi}$
        \item $\sqrt{3}\pi +\frac{3}{4}$
    \end{enumerate}

    \item Two vertices of a triangle $ABC$ are $A(3, -1)$ and $B(2, -3)$, and its orthocenter is $P(1,1)$. If the coordinates of the point $C$ are $(\alpha,\beta)$ and the center of the circle circumscribing the triangle $PAB$ is $(h,k)$, then the value of $(\alpha+\beta) + 2(h+k)$ equals:
    \begin{enumerate}
        \item $81$
        \item $15$
        \item $5$
        \item $51$
    \end{enumerate}

    \item The integral
    \begin{equation*}
\int_{\frac{1}{4}}^{\frac{3}{4}} \cos\left( 2 \cot^{-1} \sqrt{\frac{1 - x}{1 + x}} \right) \, dx
\end{equation*}

    is equal to:
    \begin{enumerate}
        \item $1/2$
        \item $1/4$
        \item $-1/4$
        \item $-1/2$
    \end{enumerate}

\item For a differentiable function $f : \mathbb{R} \to \mathbb{R}$, suppose $f'(x) = 3f(x) + a$, where $a \in \mathbb{R}$, 
$f(0) = 1$ and $\lim_{x \to \infty} f(x) = 7$. Then $9f(-\log_e 3)$ is equal to \dots.

\item Consider the circle $C: x^2 + y^2 = 4$ and the parabola $P: y^2 = 8x$. If the set of all values of $\alpha$, for which three chords of the circle $C$ on three distinct lines passing through the point $(\alpha, 0)$ are bisected by the parabola $P$ is the interval $(p, q)$, then $(2q - p)^2$ is equal to \dots.

\item If
\begin{equation*}
\left( \frac{1}{\alpha + 1} + \frac{1}{\alpha + 2} + \ldots + \frac{1}{\alpha + 1012} \right) 
- \left( \frac{1}{2 \cdot 1} + \frac{1}{4 \cdot 3} + \frac{1}{6 \cdot 5} + \ldots + \frac{1}{2024 \cdot 2023} \right) 
= \frac{1}{2024}
\end{equation*}
then $\alpha$ is equal to \dots.

\item The number of integers, between 100 and 1000 having the sum of their digits equal to 14, is \dots.

\item Consider the matrices 
\begin{equation*}
A = \myvec{ 2 & -5 \\ 3 & 20 },B=\myvec{20 \\ m}\quad and\quad X = \myvec{ x \\ y }.
\end{equation*}
Let the set of all $n$, for which the system of equations $AX = B$ has a negative solution (i.e., $x < 0$ and $y < 0$), be the interval $(a, b)$. Then 
\begin{equation*}
8\int_a^b  |A|\, dm
\end{equation*}
is equal to \dots.
\item Let $A = \{(x, y): 2x + 3y = 23, x, y \in \mathbb{N}\}$ and $B = \{x, y \in \mathbb{A}\}$. Then the number of one-one functions from $A$ to $B$ is equal to \dots.

\item Let the inverse trigonometric functions take principal values. The number of real solutions of the equation 
\begin{equation*}
2\sin^{-1} x + 3\cos^{-1} x = \frac{2\pi}{5}
\end{equation*}
is \dots.

\item Let the set of all values of $p$, for which 
\begin{equation*}
f(x) = (p^2 - 6p + 8)(\sin^2 2x - \cos^2 2x) + 2(2 - p)x + 7
\end{equation*}
does not have any critical point, be the interval $(a, b)$. Then $16ab$ is equal to \dots.

\item Let $A, B, C$ be three points on the parabola $y^2 = 6x$ and let the line segment $AB$ meet the line $L$ through $C$ parallel to the $x$-axis at the point $D$. Let $M$ and $N$ respectively be the feet of the perpendiculars from $A$ and $B$ on $L$. Then 
\begin{equation*}
\left(\frac{AM - BN}{CD}\right)^2
\end{equation*}
is equal to \dots.

\item The square of the distance of the image of the point $(6, 1, 5)$ in the line 
\begin{equation*}
\frac{x - 1}{3} = \frac{y}{2} = \frac{z - 2}{4}
\end{equation*}
from the origin is \dots.

\end{enumerate}




\end{document}

