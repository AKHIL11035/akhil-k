\let\negmedspace\undefined
\let\negthickspace\undefined
\documentclass[article]{IEEEtran}
\usepackage[a5paper, margin=10mm, onecolumn]{geometry}
%\usepackage{lmodern} % Ensure lmodern is loaded for pdflatex
\usepackage{tfrupee} % Include tfrupee package

\setlength{\headheight}{1cm} % Set the height of the header box
\setlength{\headsep}{0mm}     % Set the distance between the header box and the top of the text

\usepackage{gvv-book}
\usepackage{gvv}
\usepackage{cite}
\usepackage{amsmath,amssymb,amsfonts,amsthm}
\usepackage{algorithmic}
\usepackage{graphicx}
\usepackage{textcomp}
\usepackage{xcolor}
\usepackage{txfonts}
\usepackage{listings}
\usepackage{enumitem}
\usepackage{mathtools}
\usepackage{gensymb}
\usepackage{comment}
\usepackage[breaklinks=true]{hyperref}
\usepackage{tkz-euclide} 
\usepackage{listings}                                       
\def\inputGnumericTable{}                                 
\usepackage[latin1]{inputenc}                                
\usepackage{color}                                            
\usepackage{array}                                            
\usepackage{longtable}                                       
\usepackage{calc}                                             
\usepackage{multirow}                                         
\usepackage{hhline}                                           
\usepackage{ifthen}                                           
\usepackage{lscape}

\renewcommand{\thefigure}{\theenumi}
\renewcommand{\thetable}{\theenumi}
\setlength{\intextsep}{10pt} % Space between text and floats

\numberwithin{figure}{enumi}
\renewcommand{\thetable}{\theenumi}

% Marks the beginning of the document
\begin{document}
\bibliographystyle{IEEEtran}

\title{2012-PH-'40-52'}
\author{EE24BTECH11035 - KOTHAPALLI AKHIL}
{\let\newpage\relax\maketitle}


\begin{enumerate}
    \item The ground state wavefunction for the hydrogen atom is given by 
  \begin{equation*}
     \psi_{100} = \frac{1}{\sqrt{4 \pi}} \left(\frac{1}{a_0}\right)^{3/2} e^{-r/a_0},  
  \end{equation*}  
    where $a_0$ is the Bohr radius.

    The plot of the radial probability density, $P(r)$, for the hydrogen atom in the ground state is:
    
    \begin{enumerate}
        \item 
       

\resizebox{0.3\textwidth}{!}{%
\begin{circuitikz}
\tikzstyle{every node}=[font=\LARGE]
\draw [->, >=Stealth] (7.75,8.5) -- (15.5,8.5);
\draw [->, >=Stealth] (8,8.25) -- (8,12.75);
\draw [short] (8,11.75) .. controls (9.25,10.5) and (9.25,6.25) .. (11.5,10);
\draw [short] (11.5,10) .. controls (12,12) and (12.5,10.75) .. (13,10);
\draw [short] (13,10) .. controls (13.75,8.75) and (13.5,8.25) .. (14.75,10);
\node [font=\LARGE] at (7.25,10.75) {$P(r)$};
\node [font=\LARGE] at (10.5,8.25) {$r/a_0$};
\end{circuitikz}
}%

\label{fig:my_label}

        \item 
        \
\resizebox{0.2\textwidth}{!}{%
\begin{circuitikz}
\tikzstyle{every node}=[font=\small]
\draw (7.75,13.75) to[C] (9.75,13.75);
\node [font=\small, rotate around={-360:(0,0)}] at (7.5,13.75) {A};
\node [font=\small] at (10,13.75) {B};
\end{circuitikz}
}%

\label{fig:my_label}


        
        \item 
        \
\resizebox{0.2\textwidth}{!}{%
\begin{circuitikz}
\tikzstyle{every node}=[font=\LARGE]
\draw (7.5,12) to[D] (5.25,12);
\node [font=\normalsize] at (5,12) {A};
\node [font=\normalsize] at (7.75,12) {B};
\end{circuitikz}
}%

\label{fig:my_label}


        
        \item 
    

\resizebox{0.3\textwidth}{!}{%
\begin{circuitikz}
\tikzstyle{every node}=[font=\LARGE]
\draw [->, >=Stealth] (7.75,6.5) -- (7.75,13);
\draw [->, >=Stealth] (7.25,7.25) -- (18,7.25);
\draw [short] (7.75,7.25) .. controls (12.5,18.5) and (11.75,7.25) .. (15.75,7.25);
\node [font=\LARGE] at (6.75,9.75) {$P(r)$};
\node [font=\LARGE] at (10,6.75) {$r/a_0$};
\end{circuitikz}
}%

\label{fig:my_label}

    \end{enumerate}
    
\item Total binding energies of $^{15}\text{O}$, $^{16}\text{O}$, and $^{17}\text{O}$ are 111.96 MeV, 127.62 MeV, and 131.76 MeV, respectively. The energy gap between $1p_{1/2}$ and $1d_{5/2}$ neutron shells for the nuclei whose mass number is close to 16, is:
    
    \begin{enumerate}
        \item 4.1 MeV
        \item 11.5 MeV
        \item 15.7 MeV
        \item 19.8 MeV
    \end{enumerate}
  
    \item A particle of mass $m$ is attached to a fixed point $O$ by a weightless inextensible string of length $a$. It is rotating under the gravity as shown in the figure.
    
    The Lagrangian of the particle is
    \begin{equation*}
    L(\theta, \phi) = \frac{1}{2} ma^2 \left( \dot{\theta}^2 + \sin^2 \theta \, \dot{\phi}^2 \right) - m g a \cos \theta
    \end{equation*}
    where $\theta$ and $\phi$ are the polar angles.
    
\resizebox{0.3\textwidth}{!}{%
\begin{circuitikz}
\tikzstyle{every node}=[font=\LARGE]
\draw [dashed] (14,15) -- (14,9);
\draw [ line width=1.1pt , dashed] (14,10.25) ellipse (2.5cm and 0.75cm);
\draw [line width=1pt, ->, >=Stealth, dashed] (14,15) -- (14,15.5);
\draw [line width=1.1pt, short] (13.5,14.5) -- (14.5,14.5);
\draw [line width=1.1pt, short] (14,14.5) -- (11.5,10.25);
\node [font=\Huge] at (11.5,10.5) {.};
\draw [line width=1.1pt, ->, >=Stealth] (11.5,10.25) -- (11.5,9.25);
\draw [line width=1.1pt, ->, >=Stealth] (11.5,10.25) -- (13.25,10.25);
\draw [line width=1.1pt, <->, >=Stealth] (13.5,15) .. controls (12,15.25) and (11.75,14.25) .. (12.5,13.75);
\draw [line width=1.1pt, ->, >=Stealth, dashed] (13.25,11) -- (14.5,11);
\draw [line width=1.1pt, ->, >=Stealth, dashed] (14.25,9.5) -- (13.5,9.5);
\node [font=\Huge] at (11.5,15) {$\theta$};
\node [font=\LARGE] at (12,12.5) {a$\theta$};
\node [font=\LARGE] at (11,10.25) {m};
\node [font=\LARGE] at (11.5,9) {g};
\node [font=\LARGE] at (14.25,15.75) {z};
\end{circuitikz}
}%

\label{fig:my_label}

    The Hamiltonian of the particle is
    \begin{enumerate}
        \item 
        \begin{equation*}
        H = \frac{1}{2 m a^2} \left( p_{\theta}^2 + \frac{p_{\phi}^2}{\sin^2 \theta} \right) - m g a \cos \theta
        \end{equation*}
        \item 
        \begin{equation*}
        H = \frac{1}{2 m a^2} \left( p_{\theta}^2 + \frac{p_{\phi}^2}{\sin^2 \theta} \right) + m g a \cos \theta
        \end{equation*}
        \item 
        \begin{equation*}
        H = \frac{1}{2 m a^2} \left( p_{\theta}^2 + p_{\phi}^2 \right) - m g a \cos \theta
        \end{equation*}
        \item 
        \begin{equation*}
        H = \frac{1}{2 m a^2} \left( p_{\theta}^2 + p_{\phi}^2 \right) + m g a \cos \theta
        \end{equation*}
    \end{enumerate}

  

    \item Given $\vec{F} = \vec{r} \times \vec{B}$, where $\vec{B} = B_0 (\hat{i} + \hat{j} + \hat{k})$ is a constant vector and $\vec{r}$ is the position vector. The value of $\oint_C \vec{F} \cdot d\vec{r}$, where $C$ is a circle of unit radius centered at origin is,
    % f2.tex
\begin{figure}[!ht]
    \centering
    \resizebox{0.5\textwidth}{!}{%
        \begin{circuitikz}
            \tikzstyle{every node}=[font=\LARGE]
            \draw (5,11) rectangle (10,8.25);  % First block
            \draw (12.5,10.75) rectangle (17.5,8.25);  % Second block
            \draw (20.5,12) rectangle (23.75,7);  % Third block

            % Arrows between blocks
            \draw [->, >=Stealth] (10,9.75) -- (12.5,9.75);  % Between G1 and G2
            \draw [->, >=Stealth] (17.5,9.5) -- (20.5,9.5);  % Between G2 and 1/G3
            \draw [->, >=Stealth] (2.75,9.75) -- (4.75,9.75);  % Input arrow
            \draw [->, >=Stealth] (24,9.5) -- (26,9.5);  % Output arrow

            % Labels for input, output, and blocks
            \node [font=\Huge] at (1.75,9.75) {$Input$};
            \node [font=\Huge] at (27.5,9.5) {$Output$};
            \node [font=\LARGE] at (7.25,9.5) {$G_1$};
            \node [font=\LARGE] at (14.75,9.5) {$G_2$};
            \node [font=\LARGE] at (22,9.5) {$\frac{1}{G_3}$};
        \end{circuitikz}
    }
    
    \label{fig:my_label}
\end{figure}

    \begin{enumerate}
        \item $0$
        \item $2 \pi B_0$
        \item $-2 \pi B_0$
        \item $1$
    \end{enumerate}

    

    \item The value of the integral $\oint_C \frac{e^z}{z} \, dz$, using the contour $C$ of circle with unit radius $|z| = 1$ is
    \begin{enumerate}
        \item $0$
        \item $-2 \pi i$
        \item $1 + 2 \pi i$
        \item $2 \pi i$
    \end{enumerate}

    \item A paramagnetic system consisting of $N$ spin-half particles, is placed in an external magnetic field. It is found that $N/2$ spins are aligned parallel and the remaining $N/2$ spins are aligned antiparallel to the magnetic field. The statistical entropy of the system is
    \begin{enumerate}
        \item $2 N k_B \ln 2$
        \item $\frac{N k_B}{2} \ln 2$
        \item $\frac{3 N}{2} k_B \ln 2$
        \item $N k_B \ln 2$
    \end{enumerate}

    \item The equilibrium vibration frequency for an oscillator is observed at $2990 \, \text{cm}^{-1}$. The ratio of the frequencies corresponding to the first and the fundamental spectral lines is $1.96$. Considering the oscillator to be anharmonic, the anharmonicity constant is
    \begin{enumerate}
        \item $0.005$
        \item $0.02$
        \item $0.05$
        \item $0.1$
    \end{enumerate}

    \item At a certain temperature $T$, the average speed of nitrogen molecules in air is found to be $400 \, \text{m/s}$. The most probable and the root mean square speeds of the molecules are, respectively,
    \begin{enumerate}
        \item $355 \, \text{m/s}, \, 434 \, \text{m/s}$
        \item $820 \, \text{m/s}, \, 917 \, \text{m/s}$
        \item $152 \, \text{m/s}, \, 301 \, \text{m/s}$
        \item $422 \, \text{m/s}, \, 600 \, \text{m/s}$
    \end{enumerate}
\raggedright
\textbf{Common Data for Questions 48 and 49:}

    The wavefunction of a particle moving in free space is given by
    \begin{equation*}
    \psi = e^{ikx} + 2e^{-ikx}.
    \end{equation*}


        \item The energy of the particle is
        \begin{enumerate}
            \item $\frac{5\hbar^2 k^2}{2m}$\\
            \item $\frac{3\hbar^2 k^2}{4m}$\\
            \item $\frac{\hbar^2 k^2}{2m}$\\
            \item $\frac{\hbar^2 k^2}{m}$
        \end{enumerate}

        \item The probability current density for the real part of the wavefunction is
        \begin{enumerate}
            \item $1$
            \item $\frac{\hbar k}{m}$
            \item $\frac{\hbar k}{2m}$
            \item $0$
        \end{enumerate}

\textbf{Common Data for Questions 50 and 51:}

The dispersion relation for a one-dimensional monatomic crystal with lattice spacing $a$, which interacts via nearest neighbour harmonic potential, is given by
    \begin{equation*}
    \omega = A \left| \sin \frac{ka}{2} \right|,
    \end{equation*}
where $A$ is a constant of appropriate unit.

    
        \item The group velocity at the boundary of the first Brillouin zone is
        \begin{enumerate}
            \item $0$
            \item $1$
            \item $\frac{Aa}{2}$
            \item $\frac{1}{2} \sqrt{\frac{Aa^2}{2}}$
        \end{enumerate}

        \item The force constant between the nearest neighbour of the lattice is $(M$ is the mass of the atom$)$
        \begin{enumerate}
            \item $\frac{M A^2}{4}$
            \item $\frac{M A^2}{2}$
            \item $M A^2$
            \item $2 M A^2$
        \end{enumerate}
   
  
\textbf{Statement for Linked Answer Questions 52 and 53:}

In a hydrogen atom, consider that the electronic charge is uniformly distributed in a spherical volume of radius $a \, (= 0.5 \times 10^{-10} \, \text{m})$ around the proton. The atom is placed in a uniform electric field $\vec{E} = 30 \times 10^6 \, \text{V/m}$. Assume that the spherical distribution of the negative charge remains undistorted under the electric field.


    \item In the equilibrium condition, the separation between the positive and the negative charge centers is
    \begin{enumerate}
        \item $8.66 \times 10^{-16} \, \text{m}$
        \item $2.60 \times 10^{-15} \, \text{m}$
        \item $2.60 \times 10^{-16} \, \text{m}$
        \item $8.66 \times 10^{-15} \, \text{m}$
    \end{enumerate}
\par
\end{enumerate}

\end{document}

